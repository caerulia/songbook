\tyt{Ulice miasta (Streets of London)}
\auth{sł. i muz. Ralph McTell, tłum. Jerzy Staniucha, Jano Mościcki}

Czy widziałeś kiedyś starca na opustoszałym placu  \tab{C G a e}\\
Zniszczonym butem papier usuwa spod stóp  \tab{} \tab{F C d7 G7}\\
W oczach nie ma dumy, ręka zwisa luźno  \tab{} \tab{C G a e} \\
Na wczorajszych gazetach wczorajszy blednie druk  \tab{F C G7}\\
\hops
Ref. Więc nie mów mi, że jesteś sam  \tab{} \tab{F e G $\frac{G}{F}$ C $\frac{C}{H}$}\\
\refr I że przed tobą słońce chowa twarz  \tab{} \tab{a $\frac{a}{G}$ $\frac{D}{F}$ G G7}\\
\refr Podaj mi dłoń, pójdziemy przez ulice miasta  \tab{C G a e} \\
\refr Zobaczysz coś, co zmieni, co odmieni zdanie twe \tab{F C G7 C}\\
\hops
Czy widziałeś kobietę na ulicy miasta \\
Ubłocona jej suknia, pozlepiał włosy brud \\
Nie ma czasu na rozmowę, szybko idzie w swoją stronę \\
Dom swój z sobą taszczy w dwóch workach na chleb. \\
\hops
Ref. Więc nie mów mi, że jesteś sam\\
\refr I że przed tobą słońce chowa twarz \\
\refr Podaj mi dłoń, pójdziemy przez ulice miasta \\
\refr Zobaczysz coś, co zmieni, co odmieni zdanie twe \\
\hops
W pewnym nocnym barze, kwadrans przed północą \\
Ten sam gość przychodzi i siada bez słów \\
Jego świat się zatrzymał na brzegu filiżanki \\
Topi w niej godziny by samotnie wrócić znów \\
\hops
Ref. Więc nie mów mi, że jesteś sam\\
\refr I że przed tobą słońce chowa twarz \\
\refr Podaj mi dłoń, pójdziemy przez ulice miasta \\
\refr Zobaczysz coś, co zmieni, co odmieni zdanie twe \\
\hops
Czy widziałeś człowieka, co siedzi przed przytułkiem \\
Bo twarz mu wyblakła,  jak wstążka medalu \\
Zimą w naszym mieście śnieg sypie na szczęście \\
Zapomnianym bohaterom, o których nie dba świat \\
\hops
Ref. Więc nie mów mi, że jesteś sam\\
\refr I że przed tobą słońce chowa twarz \\
\refr Podaj mi dłoń, pójdziemy przez ulice miasta \\
\refr Zobaczysz coś, co zmieni, co odmieni zdanie twe 