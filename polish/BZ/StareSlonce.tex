\tyt{Stare Słońce}
\auth{sł. Maria Jasnorzewska-Pawlikowska, sł. i muz. Jerzy Reiser}

\begin{flushleft}
Wypieściło Stare Słońce z ziemi liść i kwiat \tab{}\tab{G CF C} \\
Malowało go na łące w modry błękit wiejskich chat \tab{d e D G} \\
Malowało go na łące w biel czystego lnu \tab{}\tab{G CF C} \\
Rozpalone stare słońce w przedwiosennym ciepłym dniu\tab{d e D G} \\
\vskip 3mm
Ref. Dało mleczom i kaczeńcom \tab{F C G C} \\
\hspace{0.9cm}i pierwiosnków żółtych wieńcom\tab{F C G C} \\
\hspace{0.9cm}U zwalonej baszty wrót \tab{}\tab{d e a} \\
\hspace{0.9cm}Swego ciepła, swoich złót \tab{}\tab{F C G C} \\
\vskip 3mm
Popatrzyło przez obłoki w przeźroczysty staw\\
Wymyśliło sobie spokój wśród zielonych mokrych traw\\
Wymyśliło sobie miłość z majowego snu \\
Rozpalone stare słońce w przedwiosennym letnim dniu\\
\vskip 3mm
Ref. Dało mleczom i kaczeńcom\\
\hspace{0.9cm}i pierwiosnków żółtych wieńcom\\
\hspace{0.9cm}U zwalonej baszty wrót\\
\hspace{0.9cm}Swego ciepła, swoich złót\\
\vskip 3mm
I tak szło po niebie modrym jak po ziemi cień \\
Radowało ludzi dobrych pieszczotami ciepłych tchnień\\
Odchodziło rozbawione do krainy snu   \\
Rozpalone stare słońce w przedwiosennym letnim dniu.\\
\vskip 3mm
Ref. Dało mleczom i kaczeńcom\\
\hspace{0.9cm}i pierwiosnków żółtych wieńcom\\
\hspace{0.9cm}U zwalonej baszty wrót\\
\hspace{0.9cm}Swego ciepła, swoich złót\\
\end{flushleft}
\clearpage
