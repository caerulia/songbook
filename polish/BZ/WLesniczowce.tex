\tyt{W leśniczówce}
\auth{sł. Konstanty Ildefons Gałczyński, muz. Jerzy Reiser}

\begin{flushleft}
\NumTabs{7}
Tu, gdzie się gwiazdy zbiegły w taką kapelę dużą, \tab{}\tab{C d$^7$ e F$^{7+}$ CE}\\
Domek z czerwonej cegły rumieni się na wzgórzu: \tab{}\tab{a F$^{7+}$ d GG$^{5}_{5+}$}\\
Za oknem las i pole, las - rozmowa sosnowa; \tab{}\tab{C d$^7$ e F$^{7+}$ C E}\\
Dzień minął i na stole stoi lampa naftowa, \tab{}\tab{a F$^{7+}$ C G C}\\
\vskip 3mm
I płynie koncert wielki \tab{}\tab{}\tab{G C}\\
Przez dęby i przez świerki - \tab{}\tab{GA d}\\
Cienie na każdej ścianie, nocne muzykowanie \tab{FA d C GC}\\
\vskip 3mm
Ref. Tańczy noc, rozśpiewała się, po pagórkach, po kotlinach \tab{G C GA d}\\
\hspace{0.9cm}Tańczy noc, roztańczyła się w wieńcu z dzikiego wina \tab{G C FC G}\\
\hspace{0.9cm}Dmie wiatr w srebrne pasma chmur a każda chmura inna \tab{G C E aG}\\
\hspace{0.9cm}Tańczy noc, roztańczyła się w wieńcu z dzikiego wina \tab{F Ca dG CF$^{7+}$ C}\\
\vskip 3mm
Wiatr chodzi  nad jeziorem trąca  dęby i graby \\
Już wieczór, a wieczorem znów zaświecamy lampy; \\
I płynie koncert wielki \\
Przez dęby i przez świerki - \\
Lamp lśnienie, migotanie, nocne muzykowanie \\
\vskip 3mm
Ref. Tańczy noc, rozśpiewała się, po pagórkach, po kotlinach\\
\hspace{0.9cm}Tańczy noc, roztańczyła się w wieńcu z dzikiego wina \\
\hspace{0.9cm}Dmie wiatr w srebrne pasma chmur a każda chmura inna \\
\hspace{0.9cm}Tańczy noc, roztańczyła się w wieńcu z dzikiego wina \\
\vskip 3mm
Gwiazdy jak śnieg się sypią, do leśniczówki wchodzą \\
Każdą okienną szybą, każdą wrześniową nocą, \\
I płynie koncert wielki \\
Przez dęby i przez świerki - \\
Księżyc na każdej ścianie, nocne muzykowanie \\
\vskip 3mm
Ref. Tańczy noc, rozśpiewała się, po pagórkach, po kotlinach\\
\hspace{0.9cm}Tańczy noc, roztańczyła się w wieńcu z dzikiego wina \\
\hspace{0.9cm}Dmie wiatr w srebrne pasma chmur a każda chmura inna \\
\hspace{0.9cm}Tańczy noc, roztańczyła się w wieńcu z dzikiego wina \\
\NumTabs{6}
\end{flushleft}
\clearpage
