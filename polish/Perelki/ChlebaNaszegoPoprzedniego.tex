\tyt{Chleba naszego poprzedniego}
\auth{sł. Wiesława Kwinto-Koczan, muz. Michał Łangowski}

A gdy mi tylko zostaną wspomnienia \tab{}\tab{a G a $\frac{a}{H} $} \\
I stopy po górach nie zechcą już nosić, \tab{}\tab{C ~G d} \\
spojrzę do tyłu, zajrzę w głąb siebie \tab{}\tab{a G d a} \\
i będę Ciebie najgorliwiej prosić: \tab{}\tab{d e F G a/G/a} \\
\hops
Chleba naszego poprzedniego daj mi, \tab{}\tab{a G a $\frac{a}{H} $ C} \\
aby nakarmić głodnych myśli sforę, \tab{}\tab{F G C $\frac{C}{H}$ a} \\
żeby z tęsknoty przestały ujadać, \tab{}\tab{F G C $\frac{C}{H}$ a} \\
u kolan przycupnęły w wieczorową porę. \tab{}\tab{F G a/G/a} \\
\hops
Chleba naszego poprzedniego daj mi,  \\
bo żyć nie umiem w żaden inny sposób,  \\
daj we wspomnieniach jego zapach poczuć  \\
pogodzić się łatwiej z zakrętami losu.  \\
\hops
Chleba naszego poprzedniego daj mi\\
aby nakarmić głodnych myśli sforę\\
żeby z tęsknoty przestały ujadać\\
u kolan przycupnęły w wieczorową porę.\\
\hops
Chleba naszego poprzedniego daj mi,  \\
bo żyć nie umiem w żaden inny sposób,  \\
daj we wspomnieniach jego zapach poczuć  \\
pogodzić się łatwiej z zakrętami losu.  \\