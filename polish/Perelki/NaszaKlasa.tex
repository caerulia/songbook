\tyt{Nasza klasa}
\auth{sł. i muz. Jacek Kaczmarski}

Co się stało z naszą klasą? Pyta Adam w Tel-Avivie \tab{d A}\\
Ciężko sprostać takim czasom, ciężko w ogóle żyć uczciwie \tab{d A}\\
Co się stało z naszą klasą? Wojtek w Szwecji w porno-klubie \tab{$F^V$ g}\\
Pisze: dobrze mi tu płacą za to, co i tak wszak lubię \tab{d A (d g A d)}\\
\hops
Kaśka z Piotrkiem są w Kanadzie, bo tam mają perspektywy \\
Staszek w Stanach sobie radzi, Paweł do Paryża przywykł \\
Gośka z Przemkiem ledwie przędą - w maju będzie trzeci bachor \\
Próżno skarżą się urzędom, że też chcieliby na Zachód \\
\hops
Za to Magda jest w Madrycie i wychodzi za Hiszpana \\
Maciek w grudniu stracił życie, gdy chodzili po mieszkaniach \\
Janusz, ten co zawiść budził, że go każda fala niesie \\
Jest chirurgiem - leczy ludzi, ale brat mu się powiesił \\
\hops
Marek siedzi za odmowę, bo nie strzelał do Michała \\
A ja piszę ich historię - i to już jest klasa cała \\
Jeszcze Filip - fizyk w Moskwie, dziś nagrody różne zbiera \\
Jeździ kiedy chce do Polski, był przyjęty przez premiera \\
\hops
Odnalazłem klasę całą - na wygnaniu, w kraju, w grobie \\
Ale coś się pozmieniało: każdy sobie żywot skrobie \\
Odnalazłem całą klasę, wyrośniętą i dojrzałą \\
Rozdrapałem młodość naszą, lecz za bardzo nie bolało \\
\hops
Już nie chłopcy, lecz mężczyźni, już kobiety, nie dziewczyny \\
Młodość szybko się zabliźni, nie ma w tym niczyjej winy \\
Wszyscy są odpowiedzialni, wszyscy mają w życiu cele \\
Wszyscy w miarę są normalni, ale przecież to niewiele \\
\hops
Nie wiem sam, co mi się marzy, jaka z gwiazd nade mną świeci \\
Gdy wśród tych nieobcych twarzy szukam ciągle twarzy dzieci \\
Czemu wciąż przez ramię zerkam, czy nie woła nikt: "kolego!" \\
Żeby ze mną zagrać w berka, lub przynajmniej w chowanego \\
\hops
Własne pędy, własne liście zapuszczamy każdy sobie \\
I korzenie oczywiście na wygnaniu, w kraju, w grobie \\
W dół, na boki, wzwyż ku słońcu, na stracenie, w prawo, w lewo \\
Kto pamięta, że to w końcu jedno i to samo drzewo... 