\tyt{Jezioro}
\auth{sł. i muz. Jan Błyszczak}

\begin{flushleft}
Tam gdzie najtęższej sośnie upić się czasem zdarzy, \tab{D e A D}\\
Gdzie wiatr, jak winem z morza jodem dmie,\tab{}\tab{h e A e} \\
Gdzie wierzba nad strumykiem rozgrzane chłodząc stopy \tab{D e A D}\\
Północ bez trudu dojrzy w ładny dzień. \tab{}\tab{h e A e}\\
Rzeźbiarz zamorski, sławny, przemierzył kiedyś lata, \tab{D h e A}\\
Wilgotnych butów znaczył każdy krok, \tab{}\tab{D h e A}\\
Zakatarzonym marszem do domu swego wracał, \tab{G D e h}\\
Zgubił cudownych luster sto. \tab{}\tab{}\tab{e D A D}\\
\vskip 3mm
Ref. W najpiękniejszym spośród luster \tab{}\tab{D A} \\
\hspace{0.9cm}zanurzyłem dzbany puste \tab{}\tab{}\tab{h fis}\\
\hspace{0.9cm}Zanim słońce cicho wzeszło na pogodę, \tab{}\tab{G D e A}\\
\hspace{0.9cm}Dzbany młodość dopijały \tab{}\tab{}\tab{e A}\\
\hspace{0.9cm}W świat odbity, świat wspaniały \tab{}\tab{D h}\\
\hspace{0.9cm}Wrysowałem swe odbicie młode. \tab{}\tab{e A h}\\
\vskip 3mm
Kiedyś już późnym latem na liście jesień spadła \\
I po swojemu splotła wystrój dnia, \\
Jak dzikich kaczek stado z odlotem się wstrzymała, \\
By w lustrze się przeglądać jeszcze raz. \\
I każdy kto przechodził u lustra mego stawał \\
Rumieńcem barw wytaczał obraz swój. \\
I tylko zima biała odbicia nie znalazła, \\
Znalazła chłodnych powiek lód. \\
\vskip 3mm
Ref. W najpiękniejszym spośród luster \\
\hspace{0.9cm}zanurzyłem dzbany puste \\
\hspace{0.9cm}Zanim słońce cicho wzeszło na pogodę, \\
\hspace{0.9cm}Dzbany młodość dopijały w świat odbity, \\
\hspace{0.9cm}świat wspaniały \\
\hspace{0.9cm}Wrysowałem swe odbicie młode. \\
\end{flushleft}
\clearpage
