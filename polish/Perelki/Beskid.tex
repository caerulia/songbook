\tyt{Beskid}
\auth{sł. i muz. Andrzej Wierzbicki}

A w Beskidzie rozzłocony buk, \tab{}\tab{G C D G}\\
A w Beskidzie rozzłocony buk. \tab{}\tab{G C G a D }\\
Będę chodził Bukowiną z dłutem w ręku,\tab{} \tab{C D G}\\
By w dziewczęcych twarzach uśmiech rzeźbić, \tab{}\tab{C G}\\
Niech nie płaczą już, \tab{}\tab{}\tab{C $\frac{C}{H}$ a D}\\
Niech się cieszą po kapliczkach moich dróg. \tab{}\tab{C D G} \\
\hops
Ref. Beskidzie, malowany cerkiewny dach,\tab{}\tab{}\tab{G C D G} \\
\refr Beskidzie, zapach miodu w bukowych pniach. \tab{}\tab{G C H7 e}\\
\refr Tutaj wracam, gdy ruda jesień na przełęcze swój tobół niesie. \tab{C D G C}\\
\refr Słucham bicia dzwonów w przedwieczorny czas. \tab{}\tab{G C $\frac{C}{H}$ a D}\\
\refr Beskidzie, malowany wiatrami dom,  \tab{}\tab{}\tab{G C D G}\\
\refr Beskidzie, tutaj słowa inaczej brzmią, \tab{}\tab{}\tab{G C H7 e}\\
\refr Kiedy krzyczę w jesienną ciszę, kiedy wiatrem szeleszczą liście, \tab{C D G C} \\
\refr Kiedy wolność się tuli w ciepło moich rąk \tab{}\tab{G C $\frac{C}{H}$ a D}\\
\refr Gdy, jak źrebak się tuli do mych rąk \tab{}\tab{}\tab{C D G}\\
\hops
A w Beskidzie zamyślony czas \\
A w Beskidzie zamyślony czas \\
Będę chodził z nim poddaszem gór \\
By zerwanych marzeń struny \\
Przywiązywać w niespokojne dłonie drzew \\
Niech mi grają na rozstajach moich dróg \\
\hops
Ref. Beskidzie, malowany cerkiewny dach, \\
\refr Beskidzie, zapach miodu w bukowych pniach. \\
\refr Tutaj wracam, gdy ruda jesień na przełęcze swój tobół niesie. \\
\refr Słucham bicia dzwonów w przedwieczorny czas. \\
\refr Beskidzie, malowany wiatrami dom, \\
\refr Beskidzie, tutaj słowa inaczej brzmią, \\
\refr Kiedy krzyczę w jesienną ciszę, kiedy wiatrem szeleszczą liście, \\
\refr Kiedy wolność się tuli w ciepło moich rąk \\
\refr Gdy, jak źrebak się tuli do mych rąk