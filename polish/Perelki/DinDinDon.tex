\tyt{Din-Din-Don}
\auth{sł. i muz. Marian Limański}

Coś w tym było, nie wiem co, że mój dziadek lubił to \tab{G G7 C}\\
Był najlepszym zegarmistrzem w okolicy \tab{}\tab{A D}\\
Każdy znał go, no bo on w samym rynku miał swój dom \tab{G G7 C G0}\\
A w nim sto zegarów biło din-din-don \tab{}\tab{G D G DG}\\
\hops
Ref. Din-din-don, din-din-don \tab{}\tab{}\tab{G G7}\\
\refr To zegarów dźwięczny ton \tab{}\tab{C D}\\
\refr W domu dziadka było ich ze sto \tab{}\tab{G G7 C G0}\\
\refr Din-din-don, w każdym kącie din-din-don \tab{G D G DG}\\
\hops
Ledwie tylko z łóżka wstał, zaraz do warsztatu gnał \\
I zabierał się do zwykłej swej roboty \\
Bardzo lubił ten swój fach, nawet mnie przyuczyć chciał \\
Aby zawsze mogły dzwonić din-din-don \\
\hops
Ref. Din-din-don, din-din-don \\
\refr To zegarów dźwięczny ton \\
\refr W domu dziadka było ich ze sto \\
\refr Din-din-don, w każdym kącie din-din-don \\
\hops
Gdy niebiański wielki dzwon mu oznajmił swym din-don \\
Że już pora wybrać się do Pana Boga \\
Wcale nie żal było mu tych zegarów jego stu \\
Przecież w niebie zegarmistrzem będzie znów \\
\hops
Ref. Din-din-don, din-din-don \\
\refr To zegarów dźwięczny ton \\
\refr W domu dziadka było ich ze sto \\
\refr Din-din-don, w każdym kącie din-din-don