\tyt{Jak}
\auth{sł. Edward Stachura, muz. Krzysztof Myszkowski}

Jak po nocnym niebie sunące białe obłoki nad lasem \tab{D A G D}\\
Jak na szyi wędrowca apaszka szamotana wiatrem \tab{e G D}\\
Jak wyciągnięte tam powyżej gwieździste ramiona wasze \\
A tu są nasze, a tu są nasze. \\
\hops
Jak suchy szloch w tę dżdżystą noc \\
Jak winny - li - niewinny sumienia wyrzut, \\
Że się żyje, gdy umarło tylu, tylu, tylu. \\
\hops
Jak suchy szloch w tę dżdżystą noc \\
Jak lizać rany celnie zadane \\
Jak lepić serce w proch potrzaskane \\
\hops
Jak suchy szloch w tę dżdżystą noc \\
Pudowy kamień, pudowy kamień \\
Jak na nim stanę, on na mnie stanie \\
On na mnie stanie, spod niego wstanę \\
\hops
Jak suchy szloch w tę dżdżystą noc \\
Jak złota kula nad wodami \\
Jak świt pod spuchniętymi powiekami \\
\hops
Jak zorze miłe, śliczne polany \\
Jak słońca pierś, jak garb swój nieść \\
Jak do was, siostry mgławicowe, ten zawodzący śpiew \\
\hops
Jak biec do końca, potem odpoczniesz, potem odpoczniesz \\
Cudne manowce, cudne manowce, cudne, cudne manowce \\