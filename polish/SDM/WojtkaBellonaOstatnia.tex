\tyt{Wojtka Bellona ostatnia ziemska podróż do Włodawy}
\auth{sł. Adam Ziemianin, muz. Krzysztof Myszkowski}

\begin{flushleft}
Z gitarą i piórem, kwietniowym wieczorem  \tab{d C} \\
jedziemy Wojtku razem do Włodawy  \tab{G} \\
Stary bieszczadnik, Majster Bieda  \tab{F G} \\
wciąż wierny górom, jak zwykle jest z nami  \tab{A} \\
I mówisz - wszystko się uda, \tab{} \tab{d F}\\
o to nie ma żadnej obawy \tab{} \tab{C A} \\
przecież jedziemy dziś \tab{} \tab{B C} \\
na dwa dni do Włodawy \tab{} \tab{d C d}\\
\vskip 3mm
W przedziale po oczach mdli mleczna żarówka  \\
mała gwiazda betlejemska  \\
I pociąg lubelski noc długą przecina  \\
buntując się na ostrych zakrętach  \\
I mówisz - wszystko będzie dobrze  \\
opowiemy im nasze sprawy  \\
przecież po to jedziemy  \\
na dwa dni do Włodawy  \\
\vskip 3mm
Z plecaka chleb wyjęty, garść soli  \\
kawałek sprytem zdobytej kiełbasy  \\
Chcemy noc przeskoczyć ciemną i niepewną \\
z biletem kupionym w nieznane  \\
Lecz mówisz - znów musi się udać  \\
choć tyle podróży za nami  \\
przecież jedziemy dziś  \\
na dwa dni do Włodawy  \\
\vskip 3mm
Dzisiaj się buntuje Czesiek król nad króle  \\
piekarz rodem z Buska wchodzi w układ  \\
Stawia pasjans z bułek i gorzej się czuje  \\
a w drewnie cierpliwym został ślad  \\
A Ty na przekór wszystkim mówisz  \\
że się uda - nie ma obawy  \\
przecież jedziemy dziś  \\
na dwa dni do Włodawy  \\
\vskip 3mm
I są wciąż pytania i są wciąż rozmowy  \\
ważne, choć tylko przedziałowe  \\
I jak z każdej wspólnej nam posiady  \\
wygląda w przyszłość zatroskany człowiek  \\
Wyciągasz wiersz ciepły jeszcze  \\
wczoraj bodajże napisany  \\
i czytasz głośno go w przedziale  \\
w naszej podróży do Włodawy  \\
\vskip 3mm
Z gitarą i piórem, kwietniowym wieczorem \\
jedziemy Wojtku razem do Włodawy \\
Stary bieszczadnik, Majster Bieda \\
wciąż wierny górom, jak zwykle jest z nami  \\
I mówisz - wszystko się uda, \\
o to nie ma żadnej obawy \\
przecież jedziemy dziś  \\
na dwa dni do Włodawy\\
\end{flushleft}
\clearpage
