\tyt{Barman}
\auth{sł. i muz. Paweł Małolepszy}

\hops
Wieczorem, przy stołach w moim barze, tym na przeciwko kościoła,\tab{C G} \\
Siadają ludzkie twarze, tych, którym Bóg pomóc nie zdołał. \tab{}\tab{G C} \\
Nad blatem błyszczącym jak czoło starego księżyca w pełni,\tab{} \tab{F C}\\
Na smutno i na wesoło spijają się frajerzy dzielni. \tab{}\tab{G C}\\
\hops
Ref. Panie Barman jeszcze nalej - nie za dużo, tak w sam raz. \tab{}\tab{F C G a}\\
\refr Nam do nieba co raz dalej, widać los nie kocha nas.\tab{} \tab{F C G G7}\\
\tab{}\tab{}\tab{}\tab{}\tab{C F C G}\\
\tab{}\tab{}\tab{}\tab{}\tab{C F C G}\\
\tab{}\tab{}\tab{}\tab{}\tab{C F C}
\hops 
Przed sumieniem się tłumaczą, że w tej właśnie szklance Whisky \\
Dziś na dnie wreszcie zobaczą swoją gwiazdę, której szukać tutaj przyszli. \\
I piją za szczęście, co przeszło gdzieś obok, nie zauważone. \\
Za nadzieję zuchwale tak grzeszną, za swe spracowane dłonie. \\
\hops
Ref. Panie Barman jeszcze nalej - nie za dużo, tak w sam raz. \\
\refr Nam do nieba co raz dalej, widać los nie kocha nas. \\
\hops
Nad ranem, gdy sprzątam w moim barze okruchy szkła - serca stłuczone, \\
Widzę odeszłe stąd twarze i słyszę ich toasty niespełnione. \\
Za miłość, którą czas zabił, za karty, co kiepsko nam idą. \\
Za tego, co nam blogosławił, za to życie, co tak nam obrzydło. \\
\hops
Ref. Panie Barman jeszcze nalej - nie za dużo, tak w sam raz. \\
\refr Nam do nieba co raz dalej, widać los nie kocha nas. \\
\refr Panie Barman jeszcze nalej - nie za dużo, tak w sam raz \\
\refr Tak by głowę mieć na karku, lecz by pamięć trafił szlag. \\
\refr Panie Barman jeszcze nalej - nie za dużo, tak w sam raz. \\
\refr Nam do nieba co raz dalej, widać los nie kocha nas. 