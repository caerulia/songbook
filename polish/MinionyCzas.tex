\tyt{Miniony czas}
\auth{sł. i muz. Krzysztof Libionko}

Dwadzieścia lat to wiek wspaniały \tab{C F}\\
To złoty wiek jak sądzi świat \tab{}\tab{G C}\\
Gdzieś z drugiej strony pola chwały \tab{C F}\\
Poległo mych dwadzieścia lat \tab{}\tab{G C}\\
To były pieskie chwile, owszem \tab{C F}\\
Lecz przeszły w mig, jak z bicza trzasł \tab{G C}\\
I sam dziś sobie ich zazdroszczę \tab{C F}\\
Umarł już, nie wróci ten czas \tab{}\tab{G $F7^+$ $F_{F-E-D}$ C}\\
\hops
Ref. Nie ma ni rys, ni skaz miniony czas \tab{}\tab{d G}\\
\refr I gdy wyciągną swe kopyta \tab{}\tab{C $\frac{C}{H}$ a}\\
\refr Wybaczyć można tym, co nie zranią już nas \tab{d G C C7 (d G $F_{F-E-D}$ C)}\\
\hops
Nie mówmy źle o nieboszczykach \\
Czy w twej pamięci małej gąski \\
Kołacze się wśród zgranych kart \\
Ów as daremnych gier miłości \\
W zatęchłym łóżku się dusiła \\
By nam powiedzieć wkrótce pas \\
A jednak z łezką ją wspominasz \\
Piękną dziś, cóż płomień jej zgasł \\
\hops
Ref. Nie ma ni rys, ni skaz miniony czas\\
\refr I gdy wyciągną swe kopyta \\
\refr Wybaczyć można tym, co nie zranią już nas\\
\hops
Nie dziwne więc że niczym płaczce \\
Rozdartych mi nie szkoda szat \\
Gdy stary szkielet w czarnej paczce \\
Odprowadzam na tamten świat \\
Choć najpodlejsza to kanalia \\
Jaką nosiła ziemia ta \\
Od płaczu niech to was nie zwalnia \\
Sczezła i czysta jest jak łza \\
\hops
Ref. Nie ma ni rys, ni skaz miniony czas\\
\refr I gdy wyciągną swe kopyta \\
\refr Wybaczyć można tym, co nie zranią już nas 