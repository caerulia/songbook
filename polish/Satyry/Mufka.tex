\tyt{Moja prześliczna mufka}
\auth{sł. i muz. trad., tłum. Jacek Kaczmarski}

Prześliczna dzieweczka na spacer raz szła \tab{C $\frac{C}{G}$ C $\frac{C}{G}$}\\
Gdy noc ją złapała wietrzysta i zła \tab{C G G7 C}\\
Być może przestraszyłby ziąb i mrok ją \tab{C $\frac{C}{G}$ C $\frac{C}{G}$}\\
Lecz miała wszak mufkę prześliczną swą \tab{C G G7 C}\\
\hops
Ref. Nie ruszaj to moje, weź precz łapy swoje,\tab{C $\frac{C}{G}$ C $\frac{C}{G}$}\\
\refr Popsujesz ją boję się o mufkę swą \tab{C G G7 C}\\
\hops
Więc szła w dół ulicy przez zamieć i mrok \\
Gdy jakiś młodzieniec z nią zrównał swój krok \\
Na widok jej uśmiech rozjaśnił mu twarz \\
Ach panno prześliczną wszak mufkę ty masz \\
\hops
Wiem dobrze, że mam mufkę śliczną jak sen \\
Co chłopców spojrzenia przyciąga co dzień \\
Różowym jedwabiem podszyta pod spodem \\
A z wierzchu futerko co chroni przed chłodem \\
\hops
Lecz mufka jest moja i nic do niej ci \\
Bo mama kazała strzec dobrze jej mi \\
Więc idź w swoją stronę i zostaw mnie już \\
Bo nie dam ci mufki i za tysiąc róż \\
\hops
Lecz noc była zimna i pannie co szła \\
Zachciało się winka kubeczek lub dwa \\
Po winku zbyt mocnym zagościł sen w niej \\
A chłopcy z jej mufką hulali że hej \\
\hops
Zbudziwszy się w płacz uderzyła i jęk \\
Zepsuli mi mufkę i szew na niej pękł \\
Odarli futerko i jedwab na fest \\
I na nic ma mufka zupełnie już jest \\
\hops
Więc młode dziewczęta strzec trzeba się wam \\
Tych chłopców co z wami chcą być sam na sam \\
Przeminą raz dwa wina szum, słodkie słówka \\
A w darze zostanie wam dziurawa mufka