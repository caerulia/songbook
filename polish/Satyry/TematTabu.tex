\tyt{Temat Tabu}
\auth{sł. Tomasz Jachimek}

Nic nie mówią o tym chłopy ani baby,\tab{C G C} \\
U nas we wsi to po prostu temat tabu \tab{C G C}\\
Nasz ksiądz proboszcz, który prawi \tab{C d}\\
Nam kazania co niedziela \tab{}\tab{G C}\\
Ma jamnika rodem prosto z Izraela \tab{C G C}\\
La la la la la la... \tab{}\tab{C G C G C}
\hops
Wątpliwości wszelkich widzu tu poniechaj \\
Bo ten jamnik wabi się Icek Mordechaj \\
Proboszcz czystej krwi katolik \\
Stara się przybliżyć Boga \\
No a trzyma żydowskiego czworonoga \\
\hops
Z nienawiścią się patrzymy tam gdzie buda, \\
Nie dość że żydowska morda, jeszcze ruda \\
Lecz w przypadku tego kundla pochodzenie to nie wszystko, \\
Bo ten jamnik jest homoseksualistą \\
\hops
Dodam jeszcze, chociaż trudno w to uwierzyć, \\
Że ten jamnik lubi jeździć na rowerze \\
No i tutaj się wydłuża \\
Psa proboszcza grzechów lista \\
Pies proboszcza - pedał, żyd, jeszcze cyklista \\
\hops
O jamniku także we wsi się powiada, \\
Że kraść lubi jabłka z sąsiedniego sadu \\
Oto kogo pod swym dachem trzyma ksiądz proboszcz dobrodziej, \\
Pedał, żyd, cyklista no i jeszcze złodziej \\
\hops
Z biegiem czasu, cóż poradzić - Boża wola, \\
Jamnik coraz wyraźniejsze ma zakola \\
Niech więc o księżowskim kundlu cały naród prawdę słyszy: \\
Pedał, żyd, cyklista, złodziej, jeszcze łysy \\
\hops
Choć to temat tabu to dośpiewam resztę, \\
Że ten jamnik jest z okolic Bukaresztu \\
No i tutaj wyliczanka niech od nowa się zaczyna: \\
Pedał, żyd, cyklista, złodziej, łysy, Cygan \\
\hops
Wbrew zakazom jamnik czyni to uparcie, \\
Z tutejszymi psami dzieli się swym żarciem \\
Pedał, żyd, cyklista, złodziej, łysy, Cygan - to nie wszystko, \\
Bo z poglądów i zachowań komunista \\
\hops
No i jeszcze jamnik ma instynkty łowcze, \\
Raz na polu zagryzł pasącą się owcę \\
Pedał, żyd, cyklista, złodziej, łysy, Cygan - boli serce, \\
No bo jeszcze komunista i morderca \\
\hops
Psia postawa niesie pozytywne skutki, \\
Niech prawdziwy temat tabu świat zobaczy \\
Nasz ksiądz proboszcz wielką słabość \\
Ma do chłopców i do wódki, \\
Lecz przy tym jamniku można mu wybaczyć