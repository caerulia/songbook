\tyt{Nuta z Ponidzia}
\auth{sł. i muz. Wojciech Bellon}

\begin{flushleft}
\tab{}\tab{}\tab{a F E ~a F E}\\
Polami, polami, po miedzach, po miedzach, \tab{a F G C$7^+$}\\
Po błocku skisłym, w mgłę i wiatr \tab{d7 G C C$7^+$}\\
Nie za szybko, kroki drobiąc \tab{}\tab{h7 E7}\\
Idzie wiosna, idzie nam \tab{}\tab{a G F E}\\
Idzie wiosna, idzie nam \tab{}\tab{a G F E a}\\
\vskip 3mm
Rozłożyła wiosna spódnicę zieloną, \\
Przykryła błota bury błam \\
Pachnie ziemia ciałem młodym \\
Póki wiosna, póki trwa \\
Póki wiosna, póki trwa \\
\vskip 3mm
Rozpuściła wiosna warkocze kwieciste \\
Zbarwniały łąki niczym kram \\
Będzie odpust pod Wiślicą \\
Póki wiosna, póki trwa \\
Póki wiosna, póki trwa \\
\vskip 3mm
Ponidzie wiosenne, Ponidzie leniwe, \tab{a F G C$7^+$} \\
Prężysz się jak do słońca kot, \tab{}\tab{d7 G C C$7^+$}\\
Rozciągnięte po tych polach, \tab{}\tab{h7 E7}\\
Lichych lasach w pstrych łozinach, \tab{h7 E7}\\
Skałkach w słońcu rozognionym, \tab{h7 E7}\\
Nidą w łąkach roziskrzoną \tab{}\tab{h7 E7}\\
Na Ponidziu wiosna trwa \tab{}\tab{a G F E}\\
Na Ponidziu wiosna trwa \tab{}\tab{a G F E}\\
Na Ponidziu...  \tab{}\tab{}\tab{a G F E a}\\
\end{flushleft}
\clearpage
