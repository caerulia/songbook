\tyt{Pejzaże Harasymowiczowskie}
\auth{sł. i muz. Wojciech Bellon}

\begin{flushleft}
Kiedy wstałem w przedświcie a Synaj \tab{G D}\\
Prawdę głosił przez trąby wiatru \tab{C e}\\
Zasmreczyły się chmur igliwiem \tab{G D}\\
Bure świerki o góry wsparte \tab{(Bure świerki)}\tab{e C D}\\
I na niebie byłem ja jeden \\
Plotąc pieśni w warkocze bukowe \\
I schodziłem na ziemię za kwestą \\
Przez skrzydlącą się bramę Lackowej (Przez Lackową)\\
\vskip 3mm
Ref. I był Beskid i były słowa \tab{}\tab{G C G}\\
\hspace{0.9cm}Zanurzone po pępki w cerkwi baniach\tab{G C D}\\
\hspace{0.9cm}Rozłożyście złotych \tab{}\tab{D}\\
\hspace{0.9cm}Smagających się z wiatrem do krwi\tab{C D G}\\
\vskip 3mm
Moje myśli biegały końmi \\
Po niebieskich mokrych połoninach \\
I modliłem się złożywszy dłonie \\
Do gór do madonny Brunatnolicej (Do Madonny)\\
A gdy serce kroplami tęsknoty \\
Jęło spadać na góry sine \\
Czarodziejskim kwiatem paproci (Czarodziejskim kwiatem)\\
Rozgwieździła się Bukowina (Bukowina)\\
\vskip 3mm
Ref. I był Beskid i były słowa\\
\hspace{0.9cm}Zanurzone po pępki w cerkwi baniach\\
\hspace{0.9cm}Rozłożyście złotych \\
\hspace{0.9cm}Smagających się z wiatrem do krwi \\
\end{flushleft}
\clearpage
