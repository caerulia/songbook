\tyt{Rzeka}
\auth{sł. i muz. Wojciech Jarociński }
\capo{4}

Wsłuchany w twą cichą piosenkę \tab{C F7+ C F7+}\\
Wyszedłem na brzeg pierwszy raz. \tab{C F7+ e}\\
Wiedziałem już rzeko, ze kocham cię rzeko \tab{F e a} \\
Że odtąd pójdę z tobą. \tab{}\tab{F e d G}\\
\hops
Ref. O, dobra rzeko, \tab{}\tab{}\tab{C F7+ C F7+}\\
\refr O mądra wodo.\tab{} \tab{}\tab{C F7+ e}\\
\refr Wiedziałaś gdzie stopy znużone prowadzić  \tab{F e a}\\
\refr Gdy sił już było brak.\tab{} \tab{}\tab{F e d G}\\
\refr Brak. \tab{}\tab{}\tab{}\tab{CF7+ C F7+}\\
\hops
Wieże miast, łuny świateł. \\
Ich oczy zszarzałe nie raz. \\
Witały mnie pustką, żegnały milczeniem \\
gdym stał się twoim nurtem \\
\hops
Ref. O, dobra rzeko,\\
\refr O mądra wodo. \\
\refr Wiedziałaś gdzie stopy znużone prowadzić \\
\refr Gdy sił już było brak. \\
\refr Brak. \\
\hops
Po dziś dzień z tobą rzeko. \\
Gdzież począł, gdzie kres dał ci Bóg. \\
Ach życia mi braknie, by szlak twój przemierzyć, \\
by poznać twą melodię.  \\
\hops
Ref. O, dobra rzeko,\\
\refr O mądra wodo. \\
\refr Wiedziałaś gdzie stopy znużone prowadzić \\
\refr Gdy sił już było brak. \\
\hops
\refr O, dobra rzeko,\\
\refr O mądra wodo. \\
\refr Wiedziałaś gdzie stopy znużone prowadzić \\
\refr Gdy sił już było brak. \\
\hop
\refr Brak. 
