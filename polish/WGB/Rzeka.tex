\tyt{Rzeka}
\auth{sł. i muz. Wojciech Jarociński }
\capo{4}

\begin{flushleft}
Wsłuchany w twą cichą piosenkę \tab{C F7+ C F7+}\\
Wyszedłem na brzeg pierwszy raz. \tab{C F7+ e}\\
Wiedziałem już rzeko, ze kocham cię rzeko \tab{F e a} \\
Że odtąd pójdę z tobą. \tab{}\tab{F e d G}\\
\vskip 3mm
Ref. O, dobra rzeko, \tab{}\tab{}\tab{C F7+ C F7+}\\
\hspace{0.9cm}O mądra wodo.\tab{} \tab{}\tab{C F7+ e}\\
\hspace{0.9cm}Wiedziałaś gdzie stopy znużone prowadzić  \tab{F e a}\\
\hspace{0.9cm}Gdy sił już było brak.\tab{} \tab{}\tab{F e d G}\\
\hspace{0.9cm}Brak. \tab{}\tab{}\tab{}\tab{CF7+ C F7+}\\
\vskip 3mm
Wieże miast, łuny świateł. \\
Ich oczy zszarzałe nie raz. \\
Witały mnie pustką, żegnały milczeniem \\
gdym stał się twoim nurtem \\
\vskip 3mm
Ref. O, dobra rzeko,\\
\hspace{0.9cm}O mądra wodo. \\
\hspace{0.9cm}Wiedziałaś gdzie stopy znużone prowadzić \\
\hspace{0.9cm}Gdy sił już było brak. \\
\hspace{0.9cm}Brak. \\
\vskip 3mm
Po dziś dzień z tobą rzeko. \\
Gdzież począł, gdzie kres dał ci Bóg. \\
Ach życia mi braknie, by szlak twój przemierzyć, \\
by poznać twą melodię.  \\
\vskip 3mm
Ref. O, dobra rzeko,\\
\hspace{0.9cm}O mądra wodo. \\
\hspace{0.9cm}Wiedziałaś gdzie stopy znużone prowadzić \\
\hspace{0.9cm}Gdy sił już było brak. \\
\vskip 3mm
\hspace{0.9cm}O, dobra rzeko,\\
\hspace{0.9cm}O mądra wodo. \\
\hspace{0.9cm}Wiedziałaś gdzie stopy znużone prowadzić \\
\hspace{0.9cm}Gdy sił już było brak. \\
\vskip 3mm
\hspace{0.9cm}Brak. \\
\end{flushleft}
\clearpage
