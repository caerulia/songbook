\tyt{Majster Bieda}
\auth{sł. i muz. Wojciech Bellon}

\tab{}\tab{}\tab{C $C^{E-G-E}$ F e d G C x2} \\
Skąd przychodził, kto go znał  \tab{}\tab{}\tab{C F}  \\
Kto mu rękę podał kiedy \tab{}\tab{}\tab{C F G} \\
Nad rowem siadał, wyjmował chleb \tab{}\tab{C G $\frac{G}{F}$} \\
Serem przekładał i dzielił się z psem \tab{}\tab{e a} \\
Tyle wszystkiego co z sobą miał  \tab{}\tab{G F e d G C} \\
Majster Bieda \\
\hops
Czapkę z głowy ściągał gdy \\
Wiatr gałęzie chylił drzewom. \\
Śmiał się do słońca i śpiewał do gwiazd. \\
Drogę bez końca co przed nim szła \\
Znał jak pięć palców, jak szeląg zły. \\
Majster Bieda \\
\hops
Nikt nie pytał skąd się wziął \\
Gdy do ognia się przysiadał \\
Wtulał się w krąg ciepła jak w kożuch \\
Znużony drogą wędrowiec boży \\
Zasypiał długo gapiąc się w noc \\
Majster Bieda \\
\hops
Aż nastąpił taki rok \\
Smutny rok tak widać trzeba \\
Nie przyszedł Bieda zieloną wiosną \\
Miejsce gdzie siadał zielskiem zarosło \\
I choć niejeden wytężał wzrok \\
Choć lato pustym gościńcem przeszło \\
Z rudymi liśćmi, jesieni schedą \\
Wiatrem niesiony popłynął w przeszłość \\
Wiatrem niesiony popłynął w przeszłość \\
Wiatrem niesiony popłynął w przeszłość \\
Majster Bieda
