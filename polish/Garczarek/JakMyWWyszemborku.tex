\tyt{Jak my w Wyszemborku Pana Jezusa witali}
\auth{sł i muz. Andrzej Garczarek}

\begin{flushleft}
Od Giżycka na Warszawę nic już nie jechało,  \tab{}\tab{a G C2 G E a} \\
Zasypało drogi wszystkie, do domu się chciało. \tab{a G C2 G E a}\\
Na przystanku Pekaesu było napisane, \tab{}\tab{a G C2 G E a}\\
Że następny (w dni robocze) idzie szósta rano... \tab{a G C2 G E a}\\
\vskip 3mm
O Wigilii pod Mrągowem - jak było, opowiem! \tab{a} \\
Bóg się rodził, moc truchlała, \tab{}\tab{}\tab{a G}\\
We wsi pies ujadał, \tab{}\tab{}\tab{E a $\frac{a}{H}$}\\
Schowaliśmy się pod daszek, \tab{}\tab{}\tab{C G}\\
Śnieg padał i padał. \tab{}\tab{}\tab{E a $\frac{a}{H}$}\\
Schowaliśmy się pod daszek, \tab{}\tab{}\tab{C G}\\
Śnieg padał i padał. \tab{}\tab{}\tab{E a}\\
\vskip 3mm
Dzięciołowski rzucił pomysł, żeby gdzie do wioski \\
Na wieczerzę się podmówić - ciągło nas do ludzi, \\
Więc my poszli na wariata, na skróty przez pole \\
Szukać miejsca w Wyszemborku przy wigilijnym stole. \\
\vskip 3mm
W Wyszemborku niestrachliwe, obcych się nie bali, \\
Jak ze swymi bielusieńkim płatkiem się łamali, \\
Tośmy siedli na przystawkę z pastorałką za przysługę, \\
Ja żem ciągnął pierwszym głosem, Dzięciołowski drugim: \\
\vskip 3mm
Gruchnęła, gruchnęła nowina w mieście, \tab{}\tab{C F G C}\\
Co żywo co żywo wszyscy pospieszcie, \tab{}\tab{C F G C}\\
Idzie gość, idzie gość - Józef z Maryją \tab{}\tab{C F G C $\frac{C}{H}$}\\
Z synaczkiem, z synaczkiem w ręku, z leliją... \tab{}\tab{a d G C}\\
\vskip 3mm
Od Giżycka na Warszawę nic już nie jechało, \\
Zasypało drogi wszystkie, do domu się chciało... \\
\end{flushleft}
\clearpage
