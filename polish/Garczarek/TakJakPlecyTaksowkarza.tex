\tyt{Tak jak plecy taksówkarza}
\auth{słowa i muzyka: Andrzej Garczarek}

Ja ci będę mówił „siostro” \tab{}\tab{G}\\
Ty mi możesz mówić „bracie”,\tab{} \tab{C G}\\
Albo „siostro” – po tej nocy nie przespanej \tab{G D} \\
Może lepiej, że nie wyszło \tab{}\tab{G G7} \\
Bo twój stary, w razie gdyby,\tab{}\tab{C G} \\
Był bezbronny, tak jak plecy taksówkarza. \tab{G D G}\\
\hops
Powiedziałaś, że go kochasz, \tab{G G7}\\
Ale czasem lubisz sobie \tab{}\tab{C G}\\
Pomyślałem to normalne, to się zdarza \tab{G D} \\
I nie mogłem się zamierzyć,\tab{}\tab{G G7} \\
Bo tak łatwo jest uderzyć \tab{}\tab{C G}\\
Z tyłu w plecy bezbronnego taksówkarza. \tab{G D G} \\
\hops
\refr Wyrównałem z nim rachunek\tab{} \tab{C G}\\
\refr Kiedy pierwszy dłoń wyciągnął \tab{}\tab{D G}\\
\refr W pięć lat po tym jak z nim poszłaś do ołtarza,\tab{A D} \\
\refr Bo powiedział, że go kochasz,\tab{} \tab{G G7} \\
\refr Ale czasem lubisz sobie\tab{}\tab{} \tab{C G} \\
\refr Prosił tylko, by nikomu nie powtarzać.\tab{} \tab{G D G} \\
\hops
Więc ja ci będę mówił „siostro” \\
Ty mi możesz mówić „bracie”, \\
Albo „siostro” – po tej nocy nie przespanej \\
Może lepiej, że nie wyszło \\
Bo twój stary, w razie gdyby, \\
Był bezbronny, tak jak plecy taksówkarza.