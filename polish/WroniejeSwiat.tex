\tyt{Wronieje Świat}
\auth{sł. i muz. Jerzy Paweł Duda, na motywach Jerzego Harasymowicza}

\begin{flushleft}
Świat wronieje już, już wronieje koń i wóz,   \tab{}\tab{}\tab{D G D} \\
W mieście zgiełk i swąd, każdy ma tam jakiś dom,  \tab{}\tab{D h E A} \\
Który z domów stu, będzie moim nie ze snu,  \tab{}\tab{}\tab{D D7 G E0} \\
Kiedy wreszcie choć jeden z domów swoim ja nazwać będę mógł \tab{D h E A D G D} \\
\vskip 3mm
Ciemny tunel drzew, słychać z dala wilczy śpiew, \\
Ćma szepnęła mi – czasem lepiej nie mieć nic. \\
Mogłem mieć już sklep, ale życie dało w łeb \\
I choć z kumpli niejeden pijak, ja świat pokruszę im jak chleb. \\
\vskip 3mm
Po cóż schodzić z gór z wysokiego konia w dół, \\
Tu wybija takt wolnej jazdy pełen szlak. \\
Ot, na niebo spójrz, ten cumulus będzie mój, \\
Niepotrzebne balkony, to jest mój rozpędzony w górach dom \\
\vskip 3mm
Świat wronieje znów, znów wronieje koń i wóz \\
Modrzewiowy trakt, znów wydyma płótno wiatr \\
Po cóż marzyć, po cóż śnić, starczą piórka wiatru trzy, \\
Zachód ziemi, niczyje runo, szumi pod krwawą łuną rdzy. \\
\end{flushleft}
\clearpage
