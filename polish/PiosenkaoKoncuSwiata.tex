\tyt{Piosenka o końcu świata}
\auth{sł. Czesław Miłosz}

W dzień końca świata  \\
Pszczoła krąży nad kwiatem nasturcji,  \\
Rybak naprawia błyszczącą sieć.  \\
Skaczą w morzu wesołe delfiny,  \\
Młode wróble czepiają się rynny  \\
I wąż ma złotą skórę, jak powinien mieć. \\
\hops
W dzień końca świata  \\
Kobiety idą polem pod parasolkami,  \\
Pijak zasypia na brzegu trawnika,  \\
Nawołują na ulicy sprzedawcy warzywa  \\
I łódka z żółtym żaglem do wyspy podpływa,  \\
Dźwięk skrzypiec w powietrzu trwa  \\
I noc gwiaździstą odmyka. \\
\hops
A którzy czekali błyskawic i gromów,  \\
Są zawiedzeni.  \\
A którzy czekali znaków i archanielskich trąb,  \\
Nie wierzą, że staje się już.  \\
Dopóki słońce i księżyc są w górze,  \\
Dopóki trzmiel nawiedza różę,  \\
Dopóki dzieci różowe się rodzą,  \\
Nikt nie wierzy, że staje się już. \\
\hops
Tylko siwy staruszek, który byłby prorokiem,  \\
Ale nie jest prorokiem, bo ma inne zajęcie,  \\
Powiada przewiązując pomidory:  \\
Innego końca świata nie będzie,  \\
Innego końca świata nie będzie. 