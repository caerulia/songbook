\tyt{Kirsten Piils Kilde}
\auth{język duński}

I gamle, længst forsvundne Tid\\
der var en deilig Pige;\\
hun var saa god, hun var saa blid,\\
hun var saa from tillige.\\
Et Crucifix i Hytten stod,\\
for det hun altid knælte,\\
naar Maanen med sit Aftenblod\\
paa Crucifixet dvælte.
\hops
Engang hun gik i denne Skov\\
og fæstede sit Öie,\\
imens hun sang sin Frelsers Lov,\\
paa Stiernen i det Höie.\\
Da foer hun vild paa spæde Fod\\
og Törsten hende plaged.\\
Tilsidst blev stakkels Pige mod,\\
hun græd og sig beklaged.
\hops
Ak Gud! hvorledes kommer jeg\\
i Aften til min Hytte?\\
Jeg reent forfeilet har min Vei,\\
jeg kan ei Foden flytte.\\
Jeg er saa törstig og her er\\
ei allermindste Draabe.\\
Kun vilde Dyr og Spögelser!\\
Ak Gud! hvad kan jeg haabe?
\hops
Og see — Hver den som troer paa Gud\\
den hielper han i Nöden! —\\
Hvor Taaren faldt et Væld sprang ud,\\
og loe i Aftenröden.\\
Da drak hun af det klare Vand\\
og fulgte Bölgen rolig.\\
Den ledte med sin Blomsterstrand\\
til Kirstens lille Bolig.
\hops
Sanct Kirsten hviler södt i Blund,\\
i Gravens Arme milde,\\
dog kommer hun hver Midnatstund\\
og stirrer paa sin Kilde.\\
Som hendes Öie den er blaae,\\
som hendes Blik den blinker.\\
Kom I! som gaae derovenpaa.\\
See, liden Kirsten vinker.\\

\clearpage
\tyt{Hevndin}

Søklandi, sníkjandi, siglandi inn\\
úr tokuni kom tað hóttandi lið.\\
Brennandi, lemjandi, sorlandi alt\\
á vegnum tað ei bleiv givið grið.\\
Rænt og stolið alt mær kært,\\
og ikki tú fært teg frá hesum vart.\\
Tá deyðans boðberi vitjar meg,\\
hann herjar um lond og kvalir teg...
\hops
\refr \sng{Niðurlag}\\
\refr Hevndin vil hava títt høvur,\\
\refr eirast skal eingin meir.\\
\refr Verndin tín veika skal vikna... nú
\hops
Stimandi, søkjando, sigladi norð\\
at finna tað útlangu stolnu hjá mær.\\
Herjandi, hevnandi, hóttandi orð\\
griðin og friðinhann ognast ei tær.\\
Oyða og brenna við bálið vil eg,\\
sum tú tá seinast vitjaði meg.\\
Hevndin skal raka tín sára rygg,\\
beinbrotin meiðslaður ligg har og hygg...
\hops
\refr \sng{Niðurlag}\\
\refr Hevndin vil hava títt høvur,\\
\refr eirast skal eingin meir.\\
\refr Verndin tín veika skal vikna... nú

\clearpage
\tyt{Jallgríms Kvæði}
\auth{sł. i muz. trad. farerskie, muz. John Áki Egholm, język farerski}

1. Eina veit eg rímuna,\\
gjørd við nýtum orðum,\\
Jallgrímur og Havgrímur\\
ráddu fyri í forðum.
\hops
\refr \sng{Niðurlag}\\
\refr Latum dans dynja dreingir,\\
\refr stásiliga stígur ringurin,\\
\refr stendur hon væl frúva.
\hops
2. Jallgrímur og Havgrímur,\\
ráddu fyri búnum brandi,\\
hin var sær á Upplondum,\\
og hin á Bjarmalandi.
\hops
3. Løgdu sær so ráðini við,\\
sum fornir gjørdu í forðum,\\
tað skal ikki kongur stýra\\
her um Heroyar norður.
\hops
4. Har komu niður á Bjarmaland\\
fimmti riddarar reystir,\\
tá svaraði Havgrímur:\\
“Hvat er tad fyri gestir?”
\hops
5. Svaraði ein av sveinunum,\\
hann smíldist undir lín:\\
“Haraldur kongur av Noregi\\
hann sendi meg til tín.
\hops
6. Haraldur hevur oss higar sent,\\
ikki við so blíðum,\\
antin skalt tú landið skatta\\
ella ímóti stríða.”
\hops
7. Havgrímur so til orða tekur,\\
mælir av tungu inna:\\
“Her hevur eingin skattur gingið\\
alt tað, ið frægir minnast.
\hops
8. Eg eri alin á Bjarmalandi,\\
giti ei landið skatta,\\
ikki skal hann Haraldur kongur\\
fregna tað og frætta.”
\hops
9. Bóru burt hans borðbúnað,\\
gjørt var av silvuri reina,\\
takið nú tann høvdinga,\\
og berið út á teinar.
\hops
10. Løgdu hendur á Havgrím unga,\\
ætlaðu út at bera,\\
hann vá upp sín bitra brand,\\
ið væl kundi brynjur skera

\clearpage
\tyt{Sigmunds Deyði}
\auth{sł. Pól Arni Holm, muz. Uni Debess, jęz. farerski}

Mítt lop fór í havið, tann myrka nátt,\\
fremmandir gestir komu á mína gátt.\\
Smildrað mín hurð,\\
Sorlað mítt heim.
\hops
\refr \sng{Niðurlag}\\
\refr Úr náttani hoyrast nú røddir:\\
\refr hvar er hann, ið háðar meg?\\
\refr Hoyr tú frændi, eitt skalt tú vita,\\
\refr lagnan skal raka teg!\\
\hops
Svimjið nú seggir og vinir, leið okkar long.\\
Vit veiddir verða sum vargar á bølniðu nátt.\\
Missa vit lív,\\
vinna vit fram.
\hops
\refr \sng{Niðurlag}\\
\refr Úr náttani hoyrast nú røddir:\\
\refr hvar er hann, ið háðar meg?\\
\refr Hoyr tú frændi, eitt skalt tú vita,\\
\refr lagnan skal raka teg!\\

\clearpage
\tyt{Feigdaferð}
\auth{sł. i muz. Pól Arni Holm}

Hoyr tú meg í dag, set teg her hoyr mítt lag.\\
Veingaði vinur mín, vilt tú vita, hvat eg kvað.\\
Ferðin mín heim, fjaldur ótti fremmand strond.\\
Flúgv tú flogið títt til fjallalandið mítt.
\hops
\refr \sng{Niðurlag}\\
\refr Alt sum eg gav, av hjarta til tín.\\
\refr Bløðir í ævir, kærleiki mín
\hops
Fríð er tín laið, frá havsins vídd til hagateig\\
Lat tey ei skarja teg vongin tín sum meg\\
Vildi eg hjálpa har, fjarskotið land mítt tað var.\\
Fjalt tokuland, í kyrrini við sand.
\hops
\refr \sng{Niðurlag}\\
\refr Alt sum eg gav, av hjarta til tín.\\
\refr Bløðir í ævir, kærleiki mín
\hops
Beri eg boð í bý, bjart brendi login mín.\\
Tel tú mína søk um harðrend fastatøk.\\
Síggi eg land og fígginda lagnufør.\\
Ravnagorr hoyrast hátt um Beinisvørð.
\hops
\refr \sng{Niðurlag}\\
\refr Alt sum eg gav, av hjarta til tín.\\
\refr Bløðir í ævir, kærleiki mín

\clearpage
\tyt{Grimmer Går På Gulvet}
\auth{sł. i muz. trad. duńskie, muz. John Áki Egholm, jęz. duński}

\sng{1} Grimmer går på gulvet, han kunne vel lege med sværd\\
I give mig Jomfru Ingeborg, udi vorherres færd\\
\hops
\refr \sng{Niðurlag}\\
\refr Nu sejler han, Grimmer af landet.
\hops
\sng{2}Mig tykkes du est så liden, du kan dig ikke omhugge\\
Hvor do kommer blandt kæmper, de drive dig alt til rygge
\hops
\sng{3}Ikke er jeg så liden, jeg kan mig fuldt vel værge\\
Når jeg slås med kæmper, mit sværd kan jeg vel røre 
\hops
\sng{4}Grimmer tager ad døren ud, alt både med harm og sorg\\
Hvad svar fik du af fader min, sagde jomfru Ingeborg
\hops
\sng{5}Det var liden Grimmer, han styrer sin snekke til land\\
Det var store Kamperen, han rækker hannem hvide hånd
\hops
\sng{6}Vi gå os på Vimminghøj og der skal striden stande\\
En af os skal livet lade førend vi af kredsen ganage
\hops
\sng{7}Det første hug som Kamperen gav, han var så gram i ord\\
Han hug til liden Grimmer, så han faldt ned til jord
\hops
\sng{8}Op stod liden Grimmer, han dvælede ikke ret længe\\
Du skal stande mig et igen, før solen går til senge
\hops
\sng{9}Det andet hug kom Grimmer til, han hug med højre hænde\\
Han hug i Kamperens forgyldte hjelm, så odden i hjertet vende
\hops
\sng{10}Det mælte Kamperen samme stund, da han haldt død til jord\\
Give det Gud fader i himmering, det vidste min broder Rådengård
\hops
\sng{11}Nu ligger store Kamperen og blodet rinder hannem til døde\\
Igen lever liden Grimmer, tager bort hand guld det røde
\hops
\sng{12}Tak have liden Grimmer, så vel holdt han sin ære\\
Månedsdagen der efter kom, lod han sit bryllup gøre


\clearpage
\tyt{Naglfar}
\auth{sł. Pól Arni Holm, muz. Uni Debess, jęz. farerski}

Lívstáður kvettist, ferð kom at enda.\\
Steypur mín drukkin í botn.\\
Hjartaband brostið, kvæðið er kvøðið,\\
takið nú árarnar inn.
\hops
\refr \sng{Niðurlag}\\
\refr Ljós mítt í lyktu brent, lív levnað en.\\
\refr Leit ei meir, tað boð er sent, deytt likam brent.\\
\refr Naglatilfar.
\hops
Skøltin mín leggja, niður at sova.\\
Gloym ei, negl mínar sker.\\
Veit ei, tí farið, viðin og borðið,\\
minnist til søguna her.
\hops
\refr \sng{Niðurlag}\\
\refr Ljós mítt í lyktu brent, lív levnað en.\\
\refr Leit ei meir, tað boð er sent, deytt likam brent.\\
\refr Naglatilfar.

\clearpage
\tyt{Síðsta Løtan}
\auth{sł. Pól Arni Holm, muz. Torbjørn Lisberg, jęz. farerski}

Kalt og oyði, sísta roynd,\\
áðrenn likam søkkur í hav.\\
Stoð tú fast á fedra grund,\\
nú tímin tín koma skal.
\hops
\refr \sng{Niðurlag}\\
\refr Veddi vind, jagstra lot,\\
\refr elti tú tínar sakir og dreymar.
\hops
Myrk og grá er oyðin strond,\\
nú langan setti síni segl.\\
Vilt tú vita, vinur mín,\\
hvussu síðsti tímin gekk?
\newpage
Døkt og kámt er andlit títt,\\
nú lagnutími stundar til.\\
Fast undir fótum er ei meir,\\
og ljósið sløkkjast vil.
\hops
\refr \sng{Niðurlag}\\
\refr Veddi vind, jagstra lot,\\
\refr elti tú tínar sakir og dreymar.

\clearpage
\tyt{Heiðin}
\auth{sł. i muz. Pól Arni Holm, jęz. farerski}

Gangið gjógnum myrkar gjáir,\\
vendi bak til vondu verð.\\
Einagongd í einsemi,\\
sólin skuggar sendi mær.
\hops
Vel tú leið á víddir vakrar,\\
val tað slóð, ið hugar tær.\\
Fótin set tú fram so fían,\\
anda tungt og hjarta slær.
\hops
\refr \sng{Niðurlag}\\
\refr Lýð tú teir fedrar fornu,\\
\refr ið gingu á tí somu leið\\
\refr Hoyr í homrum røddir teska,\\
\refr fylgja tær um vídd og teig.
\hops
Yvir høvur tað gorrið letur,\\
ravnar flúgva á skýum hátt.\\
Boð teir senda og alt teir síggja,\\
frættist skjótt við faðirs gátt.
\hops
\refr \sng{Niðurlag}\\
\refr Lýð tú teir fedrar fornu,\\
\refr ið gingu á tí somu leið\\
\refr Hoyr í homrum røddir teska,\\
\refr fylgja tær um vídd og teig.
