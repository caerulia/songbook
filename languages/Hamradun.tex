\tyt{Sneppan}
\auth{sł. Pól Arni Holm, muz. Torbjørn Lisberg, jęz. farerski}

Sólin setir seg í kvøld\\
Yvir bygd og land.\\
Grýtan kólnað, bríkin køld,\\
Ravnarnir setast á sand.\\
\hops
Ref. Í deyðans døkka dimma\\
\refr Ljós lívsins er sløkt.\\
\refr Lív í leiki dystin tapti\\
\refr Svarta deyðans økt.
\hops
Oyðin bygd við havið lá,\\
Kvirt er innan húsagátt.\\
Manntóm blikandi still er vág.\\
Døpur er tann drepandi sótt.
\hops
Eittans lív her anir enn\\
Í smáttu innan garð.\\
Nornan hon kvettir ein tráð í senn,\\
Ein enda hesin eisini fær.
\hops
Ref. Í deyðans døkka dimma\\
\refr Ljós lívsins er sløkt.\\
\refr Lív í leiki dystin tapti\\
\refr Svarta deyðans økt.
\hops
Sita skal eg sorgarvakt\\
Yvir fjøld og ætt.\\
Tá mítt síðsta orð er sagt,\\
Deyðin stíg tú innar lætt.
\hops
Ref. Í deyðans døkka dimma\\
\refr Ljós lívsins er sløkt.\\
\refr Lív í leiki dystin tapti\\
\refr Svarta deyðans økt.

\clearpage
\tyt{Snæbjørn}
\auth{sł. Pól Arni Holm, muz. trad. irlandzka, język farerski}

Snæbjørn vaknar jóladag, \tab{}\tab{e}\\
nú vetur komin er.\tab{}\tab{D}\\
Skal eg sita einsamallur her,\tab{}\tab{e}\\
kaldur og svangur í fjalsins lon,\tab{}\tab{e D G e}\\
kaldur í fjallsins lon.\tab{}\tab{e D e}
\hops
Fýra turrikløð tey komu\\
í millum meg og kong.\\
Skal eg bøta fyri mín kærleika\\
og verða settur í bolt og jarn,\\
settur í bolt og jarn.
\hops
Høgi harrin ið higar kom.\\
Hann fylla skuldi sítt trog.\\
Vildi meg at klintra í dúvurók\\
og seta mítt lív í vandaferð,\\
mítt lív í vandaferð.
\hops
Stevndur fyri oyggjating\\
fyri hjartans gávu bleiv eg\\
og dømdur eftir danalóg.\\
Men fútanum gav eg banasár\\
teir friðleysan gjørdu meg.
\hops
Jagstraður maður í oynni var eg,\\
men teir fáa meg ei,\\
í holum og rókum eg goyma meg vil.\\
Eg frælsur føddur føroyingur var,\\
frælsur føroyingur var.
\hops
Tíðliga á morgni ein bátur horvin\\
var í neystagrund.\\
Flýddur út av Skálamøl,\\
farvæl mítt land, mítt fljóð, mítt barn,\\
Snæbjørn farin er.

\clearpage
\tyt{Útlegd}
\auth{sł. i muz. Pól Arni Holm, język farerski}

Tankar mínir leita heim til tín, \tab{a G a}\\
langt burtur ert tú handan sýn.\tab{a G a}\\
Vakti mína trá, festi í mín hug,\tab{}\tab{F C G a}\\
sakni teg... nú.\tab{}\tab{}\tab{F G a}
\hops
Tá sólin setir seg í kvøld,\tab{}\tab{a G a}\\
brennur mítt hjarta, hóas náttin er køld.\tab{a G a}\\
Mítt sálarkvæði vil við lotinum senda tær \tab{F C G a}\\
eymleikans koss frá mær. \tab{}\tab{F G a}
\hops
\refr \sng{Niðurlag}\\
\refr Í dreyminum á nátt, tá mítt hjarta ókyrt slær, \tab{F C G a}\\
\refr ver tú í logn við mær...\tab{}\tab{}\tab{F G a}\\
\hops
Tá bylgjur bróta móti bergi fast,\\
harðnar sál mín, svørð mítt er hvast.\\
Rýma mátti eg, undar morð og makt\\
øði tendrað og megi vakt.
\hops
Friðleysur eg siti eina nú,\\
uttan teg, og eina ert tú.\\
Mín sorgarsang eg sendi, við alduni um hav\\
til heimland mítt, tá sól fer í kav.
\hops
\refr \sng{Niðurlag}\\
\refr Í dreyminum á nátt, tá mítt hjarta ókyrt slær,\\
\refr ver tú í logn við mær...\\

\clearpage
\tyt{Sinklars Vísa}
\auth{sł. trad. nordyckie, muz. trad. farerska, język staronordycki}

1. Herr Zinklar drog over salten Hav, \tab{d}\\
Til Norrig hans Cours monne stande;\\
Blant Guldbrands Klipper han fant sin Grav,\\
der vanked så blodig en Pande.\\
\hops
\refr \sng{Niðurlag}\\ 
\refr Vel op før Dag, \tab{}\tab{d A}\\
\refr de kommer vel over den Hede.\tab{g A d}\\
\hops
2. Herr Zinklar drog over Bølgen blaa  \tab{d F5}\\
For Svenske Penge at stride; \tab{}\tab{d }\\
Hielpe dig Gud du visselig maa \tab{A d F5}\\
I Gresset for Nordmanden bide. \tab{d }
\hops
3. Maanen skinner om Natten bleg,\\
De Vover saa sagtelig trille:\\
En Havfrue op av Vandet steeg\\
Hun spaaede Herr Zinklar ilde.
\hops
4. Vend om, vend om, du Skotske Mand!\\
Det gielder dit Liv saa fage,\\
Kommer du til Norrig, jeg siger for sand,\\
Ret aldrig du kommer tilbage.
\hops
5. Leed er din Sang, du giftige Trold!\\
Altidens du spaaer om Ulykker,\\
Fanger jeg dig en gang i Vold\\
Jeg lader dig hugge i Stykker.
\hops
6. Han seiled i Dage, han seiled i tre\\
Med alt sit hyrede Følge,\\
Den fierde Morgen han Norrig mon see,\\
Jeg vil det ikke fordølge.
\hops
7. Ved Romsdals Kyster han styred til Land\\
Erklærede sig for en Fiende;\\
Ham fulgte fiorten hundrede Mand\\
Som alle havde ondt i Sinde.
\hops
8. De skiendte og brændte hvor de drog frem,\\
Al Folket monne de krænke,\\
Oldingens Afmakt rørte ei dem,\\
De spottet den grædende Enke.
\hops
9. Barnet blev dræbt i Moderens Skiød,\\
Saa mildelig det end smiled;\\
Men Rygtet om denne Jammer og Nød\\
Til Kiernen af Landet iled.
\hops
10. Baunen lyste og Budstikken løb\\
Fra Grande til nærmeste Grande,\\
Dalens sønner i skjiul ei krøb\\
Det måtte Hr. Zinklar sande. 
\hops
11. Soldaten er ude paa Kongens Tog,\\
Vi maae selv Landet forsvare;\\
Forbandet være det Niddings Drog,\\
Som nu sit Blod vil spare!
\hops
12. De Bønder av Vaage, Lessøe og Lom,\\
Med skarpe Øxer paa Nakke\\
I Bredebøigd til sammen kom,\\
Med Skotten vilde de snakke.
\hops
13. Tæt under Lide der løber en Stie,\\
Som man monne Kringen kalde,\\
Laugen skynder sig der forbi,\\
I den skal Fienderne falde.
\hops
14. Riflen hænger ei meer paa Væg,\\
Hist sigter graahærdede Skytte,\\
Nøkken opløfter sit vaade Skiæg,\\
Og venter med Længsel sit Bytte.
\hops
15. Det første Skud Hr. Zinklar gialdt,\\
Han brøled og opgav sin Aande;\\
Hver Skotte raabte, da Obersten faldt:\\
Gud frie os af denne Vaande!
\hops
16. Frem Bønder! Frem I Norske Mænd!\\
Slaa ned, slaa ned for Fode!\\
Da ønsked sig Skotten hjem igien,\\
Han var ei ret lystig til Mode.
\hops
17. Med døde Kroppe blev Kringen strøed,\\
De Ravne fik nok at æde;\\
Det Ungdoms Blod, som her udflød,\\
De Skotske Piger begræde.
\hops
18. Ei nogen levende Siel kom hjem,\\
Som kunde sin Landsmand fortælle,\\
Hvor farligt det er at besøge dem\\
Der boe blandt Norriges Fielde.
\hops
19. End kneiser en Støtte på samme Sted,\\
Som Norges Uvenner mon true.\\
Vee hver en Nordmand, som ei bliver heed,\\
Saa tit hans Øine den skue! 

\clearpage
\tyt{Tað Er Ein Stutt Og Stokkut Løta}
\auth{sł. Mikkjal á Ryggi, muz. melodia hymnu}

1.Tað er ein stutt og stokkut løta, \\
so fara vit úr hesi verð; \\
í himmiríki aftur møta \\
vit teimum, sum vit mistu her, \\
sum tú algóði faðir mín \\
í náði leiddi heim til tín. 
\hops
2. Hjá Guði glað tey okkum bíða, \\
í sælu teimum tíðin fer. \\
O, faðir! gev oss kraft at stríða, \\
til fagnarstundin komin er, \\
tá loyst frá lívsins neyð og sorg \\
vit vinna heim í tína borg. 
\hops
3. Tá gloymd er øll vár svára møði, \\
tí Jesus turkar burt hvørt tár; \\
í endaleysu andans frøi \\
vit liva har í ótald ár; \\
tá syngja vit so glað og sæl: \\
Guð gjørdi alt so sera væl!


\clearpage
\tyt{Frísarnir}
\auth{sł i muz. Pól Arni Holm, muz. Uni Debess}

Frísaættin sterk í Føroya landi var,\\
á syðsta landsins enda setti seg.\\
Virdur á landi, virdur á havi\\
frísin var.
\hops
Men sóttin svarta legði kalda frísanna borg,\\
hon sparir ei teg, ei spari meg.\\
Renn nú skjótt, vend ei við,\\
sóttin svørt
\hops
\refr \sng{Niðurlag}\\
\refr Legg tú út í havið!\\
\refr Legg tú landi frá!\\
\refr Sett er kósin tín, legg tú landi frá!
\hops
Slepp tær undan hesi deyðans ódn,\\
ið oyðir alt livandi um teg.\\
Grúgvin kólnað, røddin tagnað,\\
sóttin svørt.
\hops
Sonur mín kom set her við mítt neyðars ból,\\
nú helheimur kaldi kallar meg.\\
Fornir frægir, renna gjøgnum sinni,\\
hoyr tú meg!
\hops
\refr \sng{Niðurlag}\\
\refr Legg tú út í havið!\\
\refr Legg tú landi frá!\\
\refr Sett er kósin tín, legg tú landi frá!
\hop 
\refr \sng{Niðurlag}\\
\refr Legg tú út í havið!\\
\refr Legg tú landi frá!\\
\refr Sett er kósin tín, legg tú landi frá!
\hops
Fækkast vit í talið, men sál okra vil\\
liva gjøgnum teg á hesum stað.\\
Brenn tú brýr, tak titt fæ,\\
far tú tín veg!

\clearpage
\tyt{Kvæðið Um Hargabrøður}

1. Eg gamal siti eina her á ársins síðstu nátt, \\
men minnini frá bragdartíð mær bera nýggjan mátt. \\
Tað er sum tit, ið fullu frá á miðjum lívsins veg, \\
nú vóru her í stovuni og tosaðu við meg. 
\hops
2. Vit vóru manga vandaferð, tá ódn av fjøllum brann \\
við kavarok og andróðri, og Røstin rísin rann;\\ 
tí ofta løgdu vit í hav, tá eingin annar var,\\
um oyggin øll í skúmi vóð, vit lendu fullvæl har.
\hops
3. Sat eg við skeytið, dragið tú, og hann við stýrisvøl, \\
ei tryggari var innanborðs í ódn á bátafjøl: \\
Av Tormansmið tá brotið bratt breyt yvir bát og mann, \\
hann snórabeint um aldurøð seg næstan tóman rann.
\hops 
4. Og hinaferð vit róðu kapp í grind ta longu leið \\
við rættar menn, vit framdu leik, og teir seg eirdu ei: \\
Vit løgdu út av Sumbiarmøl – ei linnaðu ein vørr\\
so brotið javnt við æsing stóð, – og inn á Hvalbiarfjørð.
\hops
5. Tá “Vaagen” fór ta gitnu ferð, vit vundu segl á knørr, \\
og tríati útróðrarmenn vit førdu heim um fjørð. \\
Í ódnini fell alt í fátt, men bróðir, tú tók ráð, \\
stóð róðurbundin ættmál tvey, vann inn á Havnarvág.
\hops
6. Tú, yngri bróðir, lesti teg til tops á Beinisvørð \\
um sjeyti favna meitilberg og upp á grønan svørð, \\
hálvtriðja hundrað favnar hátt tú sá í kolblátt hav, \\
tó djarvur, leyshentur tú fór ímillum loft og hav. 
\hops
7. Við línu, lesningi og stong vit intu kappabrøgd: \\
har eingin fótur troddi fyrr av okkum rás varð løgd; \\
vit tóku tjúgu favna loft og mjógvar, bert ein tá \\
á knaddar kundi krøkjast inn. Djúpt undir havið lá.
\hops
8. Vit kendu ei motor og bil, so tungt mangt takið var \\
við ár av ytstu havmiðum og burð um berg og skarð. \\
Tó gloymdu vit ei lyst og leik, ei kátan veitslufund, \\
í øllum førum kvóðu vit, sjálvt verstu vandastund.
\hops
9. Nú tosa øll um ítróttsbrøgd á vøll’ og fimleikshøll, \\
men lívið lærdi okkum leik á sjógv, um berg og fjøll. \\
Um deyðin fyri eygum stóð vit fóru sum til dans, \\
vit tonktu ei um blaðmansrós, ei steyp og laurberkrans.
